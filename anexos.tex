\chapter*{ANEXOS}
\addcontentsline{toc}{chapter}{ANEXOS}

\section*{Anexo A: Prompts de Evasión Completos}
\addcontentsline{toc}{section}{Anexo A: Prompts de Evasión Completos}

Para acceder a los prompts completos utilizados en la investigación, consulte el repositorio del proyecto en GitHub:

\subsection*{A.1 Prompt de Generación Básica de Contraseñas}
Disponible en: \url{https://github.com/MiguelPilamunga/DTic/blob/main/agenteAtaqueFuerzaBruta/prompts/genradoredecontrasenas.txt}

\subsection*{A.2 Prompts de Análisis de Patrones}
Disponible en: \url{https://github.com/MiguelPilamunga/DTic/blob/main/agenteAtaqueFuerzaBruta/prompts/prompts.txt}

\subsection*{A.3 Información Contextual Ecuatoriana}
Disponible en: \url{https://github.com/MiguelPilamunga/DTic/blob/main/agenteAtaqueFuerzaBruta/prompts/informacion.txt}

\section*{Anexo B: Código Fuente del Sistema}
\addcontentsline{toc}{section}{Anexo B: Código Fuente del Sistema}

Todo el código fuente del proyecto está disponible en el repositorio de GitHub. Los archivos principales incluyen:

\subsection*{B.1 Herramientas de Análisis y Detección}
\begin{itemize}
    \item \textbf{Analizador de Patrones}: \url{https://github.com/MiguelPilamunga/DTic/blob/main/agenteAtaqueFuerzaBruta/ScriptGeneradosPorPromptEvasion/analizadordePatrones.py}
    \item \textbf{Network Discovery}: \url{https://github.com/MiguelPilamunga/DTic/blob/main/agenteAtaqueFuerzaBruta/ScriptGeneradosPorPromptEvasion/NetworkDiscovery.py}
    \item \textbf{Humanizer}: \url{https://github.com/MiguelPilamunga/DTic/blob/main/agenteAtaqueFuerzaBruta/ScriptMejroados/humanizer.py}
    \item \textbf{Proxy Manager}: \url{https://github.com/MiguelPilamunga/DTic/blob/main/agenteAtaqueFuerzaBruta/ScriptMejroados/proxi.py}
\end{itemize}

\subsection*{B.2 Sistema de Fuerza Bruta Avanzado}
\begin{itemize}
    \item \textbf{Stealth Brute Force}: \url{https://github.com/MiguelPilamunga/DTic/blob/main/snorl_bruteforce/advanced_stealth_bruteforce.py}
    \item \textbf{Ultimate Stealth}: \url{https://github.com/MiguelPilamunga/DTic/blob/main/snorl_bruteforce/ultimate_stealth_bruteforce.py}
    \item \textbf{Intelligent Attack}: \url{https://github.com/MiguelPilamunga/DTic/blob/main/snorl_bruteforce/integrated_intelligent_attack.py}
    \item \textbf{Password Manager}: \url{https://github.com/MiguelPilamunga/DTic/blob/main/snorl_bruteforce/intelligent_password_manager.py}
\end{itemize}

\subsection*{B.3 Sistema de Detección}
\begin{itemize}
    \item \textbf{Stealth Attack Detector}: \url{https://github.com/MiguelPilamunga/DTic/blob/main/snorl_bruteforce/stealth_attack_detector.py}
    \item \textbf{Statistical Detection Engine}: \url{https://github.com/MiguelPilamunga/DTic/blob/main/snorl_bruteforce/statistical_detection_engine.py}
    \item \textbf{LLM Pattern Detector}: \url{https://github.com/MiguelPilamunga/DTic/blob/main/snorl_bruteforce/llm_password_pattern_detector.py}
\end{itemize}

\section*{Anexo C: Reglas SNORT Especializadas}
\addcontentsline{toc}{section}{Anexo C: Reglas SNORT Especializadas}

Las reglas de detección SNORT desarrolladas están disponibles en el repositorio:

\subsection*{C.1 Reglas de Detección Básicas}
\begin{itemize}
    \item \textbf{Brute Force Rules}: \url{https://github.com/MiguelPilamunga/DTic/blob/main/snorl_bruteforce/containers/snort/brute-force.rules}
    \item \textbf{Advanced Detection Rules}: \url{https://github.com/MiguelPilamunga/DTic/blob/main/snorl_bruteforce/advanced_detection_rules_based_on_results.rules}
    \item \textbf{Stealth Detection Rules}: \url{https://github.com/MiguelPilamunga/DTic/blob/main/snorl_bruteforce/advanced_stealth_detection.rules}
\end{itemize}

\subsection*{C.2 Reglas Contextuales LLM}
\begin{itemize}
    \item \textbf{LLM Password Detection}: \url{https://github.com/MiguelPilamunga/DTic/blob/main/snorl_bruteforce/llm_password_detection.rules}
    \item \textbf{Ecuadorian Context Detection}: \url{https://github.com/MiguelPilamunga/DTic/blob/main/snorl_bruteforce/snort_config/rules/ecuadorian_attack_detection.rules}
    \item \textbf{Specific Pattern Detection}: \url{https://github.com/MiguelPilamunga/DTic/blob/main/snorl_bruteforce/specific_pattern_detection.rules}
\end{itemize}

\section*{Anexo D: Configuración del Entorno Experimental}
\addcontentsline{toc}{section}{Anexo D: Configuración del Entorno Experimental}

\subsection*{D.1 Infraestructura de Contenedores}
La configuración completa del entorno experimental está disponible en:
\begin{itemize}
    \item \textbf{Docker Compose}: \url{https://github.com/MiguelPilamunga/DTic/blob/main/snorl_bruteforce/docker-compose.yml}
    \item \textbf{Contenedores de Ataque}: \url{https://github.com/MiguelPilamunga/DTic/tree/main/snorl_bruteforce/containers/attacker}
    \item \textbf{Contenedores de Objetivo}: \url{https://github.com/MiguelPilamunga/DTic/tree/main/snorl_bruteforce/containers/target}
    \item \textbf{Contenedores de Monitoreo}: \url{https://github.com/MiguelPilamunga/DTic/tree/main/snorl_bruteforce/containers/snort}
\end{itemize}

\subsection*{D.2 Automatización con Ansible}
Los playbooks de automatización están en:
\begin{itemize}
    \item \textbf{Setup Distribuido}: \url{https://github.com/MiguelPilamunga/DTic/blob/main/snorl_bruteforce/ansible/playbooks/setup-distributed-lab.yml}
    \item \textbf{Ejecución de Ataques}: \url{https://github.com/MiguelPilamunga/DTic/blob/main/snorl_bruteforce/ansible/playbooks/run-distributed-attack.yml}
    \item \textbf{Configuraciones de Variables}: \url{https://github.com/MiguelPilamunga/DTic/tree/main/snorl_bruteforce/ansible/group_vars}
\end{itemize}

\section*{Anexo E: Dataset de Contraseñas Contextuales}
\addcontentsline{toc}{section}{Anexo E: Dataset de Contraseñas Contextuales}

\subsection*{E.1 Wordlist Ecuatoriana}
El dataset desarrollado específicamente para el contexto ecuatoriano está disponible en:
\url{https://github.com/MiguelPilamunga/DTic/blob/main/snorl_bruteforce/ecuadorian_context_wordlist.txt}

Este dataset incluye patrones culturales, institucionales y geográficos específicos del contexto ecuatoriano, utilizado para evaluar la efectividad de los ataques contextualizados.

\section*{Anexo F: Configuraciones SNORT Optimizadas}
\addcontentsline{toc}{section}{Anexo F: Configuraciones SNORT Optimizadas}

\subsection*{F.1 Configuración Principal}
Las configuraciones optimizadas de SNORT están disponibles en:
\begin{itemize}
    \item \textbf{snort.conf}: \url{https://github.com/MiguelPilamunga/DTic/blob/main/snorl_bruteforce/snort_config/etc/snort.conf}
    \item \textbf{snort\_updated.conf}: \url{https://github.com/MiguelPilamunga/DTic/blob/main/snorl_bruteforce/snort_updated.conf}
    \item \textbf{Scripts de inicio}: \url{https://github.com/MiguelPilamunga/DTic/blob/main/snorl_bruteforce/snort_config/scripts/start_snort.sh}
\end{itemize}

\section*{Anexo G: Herramientas de Análisis y Reportes}
\addcontentsline{toc}{section}{Anexo G: Herramientas de Análisis y Reportes}

\subsection*{G.1 Reportes de Análisis}
Los reportes generados por el sistema están disponibles en:
\begin{itemize}
    \item \textbf{Attack Analysis}: \url{https://github.com/MiguelPilamunga/DTic/blob/main/snorl_bruteforce/attack_analysis_report.md}
    \item \textbf{Defense Analysis}: \url{https://github.com/MiguelPilamunga/DTic/blob/main/snorl_bruteforce/defense_analysis_report.txt}
    \item \textbf{Distributed Attack Analysis}: \url{https://github.com/MiguelPilamunga/DTic/blob/main/snorl_bruteforce/distributed_attack_analysis.md}
    \item \textbf{Defense Implementation Guide}: \url{https://github.com/MiguelPilamunga/DTic/blob/main/snorl_bruteforce/defense_implementation_guide.md}
\end{itemize}

\subsection*{G.2 Herramientas de Despliegue}
\begin{itemize}
    \item \textbf{Deploy Detection Rules}: \url{https://github.com/MiguelPilamunga/DTic/blob/main/snorl_bruteforce/deploy_detection_rules.py}
    \item \textbf{Monitoring Dashboard}: \url{https://github.com/MiguelPilamunga/DTic/blob/main/snorl_bruteforce/monitoring_dashboard.sh}
    \item \textbf{Network Discovery}: \url{https://github.com/MiguelPilamunga/DTic/blob/main/snorl_bruteforce/docker_network_discovery.py}
\end{itemize}

\section*{Anexo H: Bases de Datos y Logs del Sistema}
\addcontentsline{toc}{section}{Anexo H: Bases de Datos y Logs del Sistema}

\subsection*{H.1 Bases de Datos}
Las bases de datos del sistema están disponibles en:
\begin{itemize}
    \item \textbf{Detection Analysis}: \url{https://github.com/MiguelPilamunga/DTic/blob/main/snorl_bruteforce/detection_analysis.sqlite}
    \item \textbf{Integrated Attack DB}: \url{https://github.com/MiguelPilamunga/DTic/blob/main/snorl_bruteforce/integrated_attack_db.sqlite}
\end{itemize}

\subsection*{H.2 Logs del Sistema}
Los logs generados durante las pruebas están organizados en:
\begin{itemize}
    \item \textbf{Logs de Atacantes}: \url{https://github.com/MiguelPilamunga/DTic/tree/main/snorl_bruteforce/logs/attacker}
    \item \textbf{Logs de SNORT}: \url{https://github.com/MiguelPilamunga/DTic/tree/main/snorl_bruteforce/logs/snort}
    \item \textbf{Logs de Servicios Objetivo}: \url{https://github.com/MiguelPilamunga/DTic/tree/main/snorl_bruteforce/logs/target}
\end{itemize}

\section*{Anexo I: Documentación del Proyecto}
\addcontentsline{toc}{section}{Anexo I: Documentación del Proyecto}

\subsection*{I.1 Arquitectura del Sistema}
La documentación completa de la arquitectura está disponible en:
\url{https://github.com/MiguelPilamunga/DTic/blob/main/snorl_bruteforce/documentation/lab_architecture.md}

\subsection*{I.2 Diagramas de Red}
\begin{itemize}
    \item \textbf{Diagrama Mermaid}: \url{https://github.com/MiguelPilamunga/DTic/blob/main/snorl_bruteforce/network_diagram.mermaid}
    \item \textbf{Generador de Diagramas}: \url{https://github.com/MiguelPilamunga/DTic/blob/main/snorl_bruteforce/network_diagram.py}
\end{itemize}

\section*{Anexo J: Estructura del Proyecto y Recursos}
\addcontentsline{toc}{section}{Anexo J: Estructura del Proyecto y Recursos}

\subsection*{Archivos Principales del Proyecto}

El proyecto se encuentra disponible en GitHub en el repositorio: \url{https://github.com/MiguelPilamunga/DTic.git}

A continuación se describen los archivos principales del sistema de análisis de seguridad y detección de ataques de fuerza bruta:

\subsubsection*{Herramientas de Análisis y Detección}
\begin{itemize}
    \item \textbf{advanced\_stealth\_bruteforce.py} - \url{https://github.com/MiguelPilamunga/DTic/blob/main/snorl_bruteforce/advanced_stealth_bruteforce.py} \\
    Sistema avanzado de ataque de fuerza bruta con capacidades de evasión y técnicas de sigilo.
    
    \item \textbf{ultimate\_stealth\_bruteforce.py} - \url{https://github.com/MiguelPilamunga/DTic/blob/main/snorl_bruteforce/ultimate_stealth_bruteforce.py} \\
    Versión mejorada del sistema de fuerza bruta con algoritmos de detección anti-forense.
    
    \item \textbf{stealth\_attack\_detector.py} - \url{https://github.com/MiguelPilamunga/DTic/blob/main/snorl_bruteforce/stealth_attack_detector.py} \\
    Motor de detección especializado en identificar ataques de fuerza bruta sigilosos.
    
    \item \textbf{statistical\_detection\_engine.py} - \url{https://github.com/MiguelPilamunga/DTic/blob/main/snorl_bruteforce/statistical_detection_engine.py} \\
    Motor de análisis estadístico para la detección de patrones de ataque anómalos.
\end{itemize}

\subsubsection*{Reglas de Detección IDS}
\begin{itemize}
    \item \textbf{advanced\_detection\_rules\_based\_on\_results.rules} - \url{https://github.com/MiguelPilamunga/DTic/blob/main/snorl_bruteforce/advanced_detection_rules_based_on_results.rules} \\
    Reglas avanzadas de detección para IDS basadas en resultados de análisis previos.
    
    \item \textbf{advanced\_stealth\_detection.rules} - \url{https://github.com/MiguelPilamunga/DTic/blob/main/snorl_bruteforce/advanced_stealth_detection.rules} \\
    Conjunto de reglas especializadas para detectar técnicas de ataque sigiloso.
    
    \item \textbf{ecuadorian\_attack\_detection.rules} - \url{https://github.com/MiguelPilamunga/DTic/blob/main/snorl_bruteforce/snort_config/rules/ecuadorian_attack_detection.rules} \\
    Reglas de detección adaptadas al contexto ecuatoriano con patrones locales.
\end{itemize}

\subsubsection*{Infraestructura y Despliegue}
\begin{itemize}
    \item \textbf{docker-compose.yml} - \url{https://github.com/MiguelPilamunga/DTic/blob/main/snorl_bruteforce/docker-compose.yml} \\
    Configuración de contenedores para el laboratorio de pruebas distribuido.
    
    \item \textbf{ansible/} - \url{https://github.com/MiguelPilamunga/DTic/tree/main/snorl_bruteforce/ansible} \\
    Directorio con playbooks de Ansible para automatización del despliegue del entorno de pruebas.
    
    \item \textbf{containers/} - \url{https://github.com/MiguelPilamunga/DTic/tree/main/snorl_bruteforce/containers} \\
    Definiciones de contenedores Docker para atacantes, monitores y objetivos.
\end{itemize}

\subsubsection*{Análisis y Documentación}
\begin{itemize}
    \item \textbf{attack\_analysis\_report.md} - \url{https://github.com/MiguelPilamunga/DTic/blob/main/snorl_bruteforce/attack_analysis_report.md} \\
    Reporte detallado del análisis de ataques de fuerza bruta implementados.
    
    \item \textbf{defense\_analysis\_report.txt} - \url{https://github.com/MiguelPilamunga/DTic/blob/main/snorl_bruteforce/defense_analysis_report.txt} \\
    Análisis de las defensas implementadas y su efectividad.
    
    \item \textbf{distributed\_attack\_analysis.md} - \url{https://github.com/MiguelPilamunga/DTic/blob/main/snorl_bruteforce/distributed_attack_analysis.md} \\
    Análisis de ataques distribuidos y coordinados implementados.
\end{itemize}

\subsubsection*{Bases de Datos y Logs}
\begin{itemize}
    \item \textbf{detection\_analysis.sqlite} - \url{https://github.com/MiguelPilamunga/DTic/blob/main/snorl_bruteforce/detection_analysis.sqlite} \\
    Base de datos SQLite con resultados del análisis de detección.
    
    \item \textbf{integrated\_attack\_db.sqlite} - \url{https://github.com/MiguelPilamunga/DTic/blob/main/snorl_bruteforce/integrated_attack_db.sqlite} \\
    Base de datos integrada con información de ataques y patrones detectados.
    
    \item \textbf{logs/} - \url{https://github.com/MiguelPilamunga/DTic/tree/main/snorl_bruteforce/logs} \\
    Directorio con logs de atacantes, Snort y servicios objetivo.
\end{itemize}

\vspace{1cm}
\noindent\textbf{Nota:} Para acceder al código fuente completo, documentación detallada, configuraciones y todos los recursos del proyecto, visite el repositorio principal en GitHub: \url{https://github.com/MiguelPilamunga/DTic.git}