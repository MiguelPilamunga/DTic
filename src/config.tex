\usepackage[activeacute,spanish]{babel}
\usepackage[utf8]{inputenc}
\usepackage[T1]{fontenc}
\usepackage[a4paper,left=3cm,top=3cm,right=2.5cm,bottom=2.5cm,footskip=1.5cm]{geometry} 
\usepackage{datetime}
\usepackage{amsmath}
\usepackage{amssymb}
\usepackage{amsfonts}
\usepackage{tcolorbox}
\tcbuselibrary{skins}
\usepackage{verbatim}
\usepackage{fancyvrb}
\usepackage{fancybox}
\usepackage{framed}

% ============================================
% PAQUETES PARA CORRECCIÓN DE POSICIONAMIENTO DE FIGURAS
% ============================================
\usepackage{float}
\usepackage{placeins}
\usepackage{wrapfig}

% Configuración para mejorar el posicionamiento de floats
\renewcommand{\topfraction}{0.9}
\renewcommand{\bottomfraction}{0.8}
\setcounter{topnumber}{2}
\setcounter{bottomnumber}{2}
\setcounter{totalnumber}{4}
\renewcommand{\textfraction}{0.07}
\renewcommand{\floatpagefraction}{0.7}

\usepackage{url}
\newdateformat{epnDate}{\monthname[\THEMONTH] \THEYEAR} 
\spanishdecimal{.}
\usepackage[style=ieee, backend=bibtex]{biblatex}
\addbibresource{bibliografia.bib}
\usepackage{csquotes}
\usepackage{url}
\urlstyle{same}
\usepackage{helvet}
\renewcommand{\familydefault}{\sfdefault}
\linespread{1.5}
\newcommand{\sizeveinticuatro}{\fontsize{24pt}{20pt}\selectfont}
\newcommand{\sizedieciseis}{\fontsize{16pt}{20pt}\selectfont} 
\newcommand{\sizecatorce}{\fontsize{14}{20pt}\selectfont}
\newcommand{\sizedoce}{\fontsize{12}{20pt}\selectfont}
\setlength\parindent{0pt}
\usepackage{fancyhdr}
\pagestyle{fancyplain}
\fancyhf{}
\fancyfoot[C]{\thepage}
\renewcommand{\headrulewidth}{0pt}
\geometry{headheight=15pt}
\usepackage{setspace}
\flushbottom
\setlength{\footnotesep}{0.4cm}
\setlength{\skip\footins}{1.1cm}
\usepackage{ragged2e}
\usepackage{scrextend}
\deffootnotemark{\textsuperscript{[\thefootnotemark]}}
\deffootnote[3em]{3em}{0em}{\parbox[b][\height][r]{2.3em}{\footnotesize\textsuperscript{[\thefootnotemark]}}\enskip}
\usepackage{titlesec}
\titleformat{\chapter}[hang]{\bfseries\sizedieciseis}{\MakeUppercase{}\ \thechapter}{5.0mm}{\sizedieciseis\MakeUppercase}
\titleformat{\section}[hang]{\bfseries\sizecatorce}{\thesection}{5.0mm}{\sizecatorce\MakeUppercase}
\titleformat{\subsection}[hang]{\bfseries\sizecatorce}{\thesubsection}{5.0mm}{\sizecatorce}
\titleformat{\subsubsection}[hang]{\bfseries\sizedoce}{\thesubsubsection}{5.0mm}{\sizedoce}
\titleformat{\paragraph}[hang]{\em\sizedoce}{\theparagraph}{5.0mm}{\sizedoce}
\newcommand{\titulos}{\sf\bf\sizecatorce\centerline}
\newcommand{\titulosizq}{\sf\bf\sizecatorce}
\setcounter{secnumdepth}{3}
\setcounter{tocdepth}{3}

\usepackage[breaklinks]{hyperref} 
\hypersetup{pdfauthor={Luis Miguel Pilamunga Agualsaca},pdftitle={Herramientas de Seguridad Ofensiva},pdfsubject={Evaluación de la Detección de Ataques de Fuerza Bruta Generados por IA con SNORT},pdfkeywords={Seguridad, Ofensiva, IA, SNORT},colorlinks,citecolor=black,filecolor=black,linkcolor=black,urlcolor=black}
\usepackage[all]{hypcap}
\usepackage[titles]{tocloft}
\setlength\cftbeforetoctitleskip{0pt}
\setlength\cftaftertoctitleskip{1cm}
\renewcommand\cftchappresnum{\chaptername\space}
\renewcommand\cftchappresnum{ }
\setlength{\cftchapnumwidth}{2em}
\newcommand\centrarcelda[1]{\let\temp=\\#1\let\\=\temp}
\usepackage{array}
\usepackage{calc}
\usepackage[table]{xcolor}
\usepackage{booktabs}
\usepackage{tabulary}
\usepackage{longtable}
\usepackage{setspace}
\setlength\tymin{5cm}
\usepackage{multirow}
\definecolor{bluetable}{RGB}{175,198,233}
\arrayrulecolor{bluetable}
\setlength{\arrayrulewidth}{.9pt}
\usepackage{hhline}
\usepackage{etoolbox}
\AtBeginEnvironment{longtable}{\small}
\AtBeginEnvironment{tabular}{\small}
\usepackage{pdflscape}

% ============================================
% CONFIGURACIÓN CORREGIDA DE LISTINGS
% ============================================
\usepackage{listings}
\usepackage{color}

% Definir todos los colores necesarios
\definecolor{letraAzul}{cmyk}{1,0.5,0,0.5}
\definecolor{lstrule}{RGB}{158,180,204}
\definecolor{fondo}{RGB}{245,245,250}
\definecolor{gray}{rgb}{0.5,0.5,0.5}
\definecolor{darkviolet}{rgb}{0.5,0,0.4}
\definecolor{darkpink}{rgb}{0.8,0.3,0.5}
\definecolor{codegreen}{rgb}{0,0.6,0}
\definecolor{codegray}{rgb}{0.5,0.5,0.5}
\definecolor{codepurple}{rgb}{0.58,0,0.82}
\definecolor{backcolour}{rgb}{0.95,0.95,0.92}

% Configuración base SIN lenguaje específico
\lstset{
    basicstyle=\footnotesize\ttfamily\color{letraAzul},
    numbers=left,
    numberstyle=\scriptsize\color{gray},
    numberfirstline=true,
    firstnumber=1,
    stepnumber=5,
    numbersep=8pt,
    backgroundcolor=\color{fondo},
    showspaces=false,
    showstringspaces=false,
    showtabs=false,
    frame=single,
    rulecolor=\color{lstrule},
    tabsize=4,
    captionpos=t,
    breaklines=true,
    breakatwhitespace=false,
    title=\lstname,
    keywordstyle=\bfseries\color{darkviolet},
    commentstyle=\color{gray},
    stringstyle=\color{darkpink},
    escapeinside={\%*}{*)},
    inputencoding=utf8,
    emphstyle=\color{red},
    extendedchars=true,
    literate={á}{{\'a}}1{é}{{\'e}}1{í}{{\'i}}1{ó}{{\'o}}1{ú}{{\'u}}1{ñ}{{\~n}}1
}

% Definir estilos específicos seguros
\lstdefinestyle{mystyle}{
    backgroundcolor=\color{backcolour},   
    commentstyle=\color{codegreen},
    keywordstyle=\color{magenta},
    numberstyle=\tiny\color{codegray},
    stringstyle=\color{codepurple},
    basicstyle=\ttfamily\footnotesize,
    breakatwhitespace=false,         
    breaklines=true,                 
    captionpos=b,                    
    keepspaces=true,                 
    numbers=left,                    
    numbersep=5pt,                  
    showspaces=false,                
    showstringspaces=false,
    showtabs=false,                  
    tabsize=2
}

\lstdefinestyle{python}{
    language=Python,
    style=mystyle
}

\lstdefinestyle{bash}{
    language=bash,
    style=mystyle
}

\lstdefinestyle{plaintext}{
    language={},
    style=mystyle
}

\lstdefinestyle{yaml}{
    language={},
    style=mystyle,
    morekeywords={version, services, build, container_name, networks, ports, volumes, environment, image, depends_on}
}

% ============================================
% PAQUETES PARA GRÁFICOS E IMÁGENES
% ============================================
\usepackage{graphicx,type1cm,eso-pic}
\usepackage[font=small,format=plain,labelfont=bf,up,justification=default,compatibility=false]{caption}
\usepackage{subcaption}
\usepackage{datatool}

% ============================================
% GLOSARIOS Y NOMENCLATURA
% ============================================
\usepackage[toc,acronym,xindy]{glossaries}
\usepackage{mfirstuc}
\renewcommand{\glsnamefont}[1]{\makefirstuc{#1}}
\usepackage{glossary-super}
\makeglossaries

% ============================================
% ENUMERACIONES Y LISTAS
% ============================================
\usepackage{enumitem}
\newcommand\litem[1]{\item{\bfseries #1\enspace}}
\newenvironment{myindent}[1]{\begin{list}{}{\setlength{\leftmargin}{#1}}\item[]}{\end{list}}
\usepackage{pifont}
\renewcommand{\labelitemi}{\ding{112}}
\renewcommand{\labelitemii}{\ding{71}}

% ============================================
% COMANDOS PERSONALIZADOS PARA REFERENCIAS
% ============================================
\newcommand{\fullrefuno}[1]{(véase \ref{#1}, pág. \pageref{#1})}
\newcommand{\fullref}[2]{en \ref{#1}, pág. \pageref{#2}}
\newcommand{\refdos}[2]{(véase \ref{#1} y \ref{#2})}
\newcommand{\fullreffig}[1]{Fig.~\ref{#1} pág. \pageref{#1}} 
\newcommand{\fullreftab}[1]{Tab.~\ref{#1} pág. \pageref{#1}} 
\newcommand{\fullrefcod}[1]{Cód.~\ref{#1} pág. \pageref{#1}} 
\newcommand{\fullrefanx}[1]{(véase el Anexo~\ref{#1})} 

% ============================================
% CONFIGURACIÓN DE ANEXOS
% ============================================
\hyphenation{sa-ffer soft-ware Jesse Nielsen}
\usepackage[toc,title,header]{appendix}
\renewcommand{\appendixname}{Anexo}
\renewcommand{\spanishappendixname}{Anexo}
\renewcommand{\appendixtocname}{Anexos}
\renewcommand{\appendixpagename}{Anexos}
\makeatletter
\newcommand*\updatechaptername{\addtocontents{toc}{\protect\renewcommand*\protect\cftchappresnum{\@chapapp\ }\setlength{\cftchapnumwidth}{5em}}}
\makeatother
\makeatletter
\appto{\appendices}{\def\Hy@chapapp{Appendix}}
\makeatother

% ============================================
% COMANDOS ADICIONALES PERSONALIZADOS
% ============================================
\newcommand{\seccion}[1]{\section{#1}}
\newcommand{\subseccion}[1]{\subsection{#1}}
\newcommand{\subsubseccion}[1]{\subsubsection{#1}}

% ============================================
% PAQUETES PARA GRÁFICAS CON PGF/TIKZ
% ============================================
\usepackage{pgfplots}
\usepackage{tikz}
\pgfplotsset{compat=1.18}
\usetikzlibrary{patterns}

% Configurar pgfplots para gráficas de la metodología
\pgfplotsset{
    methodology style/.style={
        width=12cm,
        height=8cm,
        grid=major,
        grid style={dashed, gray!30},
        legend style={at={(0.02,0.98)}, anchor=north west, font=\small},
        xlabel style={font=\small},
        ylabel style={font=\small},
        tick label style={font=\scriptsize},
        every axis plot/.append style={thick}
    }
}