\usepackage{natbib}
\usepackage[activeacute,spanish]{babel}
\usepackage[utf8]{inputenc}
\usepackage[T1]{fontenc}
\usepackage[a4paper,left=3cm,top=3cm,right=2.5cm,bottom=2.5cm,footskip=1.5cm]{geometry} 
\usepackage{datetime}
\newdateformat{epnDate}{\monthname[\THEMONTH] \THEYEAR} 
\spanishdecimal{.}
\usepackage[style=ieee]{biblatex}
\addbibresource{bibliografia.bib}
\usepackage{csquotes}
\usepackage{url}
\urlstyle{same}
\usepackage{helvet}
\renewcommand{\familydefault}{\sfdefault}
\linespread{1.5}
\newcommand{\sizeveinticuatro}{\fontsize{24pt}{20pt}\selectfont}
\newcommand{\sizedieciseis}{\fontsize{16pt}{20pt}\selectfont} 
\newcommand{\sizecatorce}{\fontsize{14}{20pt}\selectfont}
\newcommand{\sizedoce}{\fontsize{12}{20pt}\selectfont}
\setlength\parindent{0pt}
\usepackage{fancyhdr}
\pagestyle{fancyplain}
\fancyhf{}
\fancyfoot[C]{\thepage}
\renewcommand{\headrulewidth}{0pt}
\geometry{headheight=15pt}
\usepackage{setspace}
\flushbottom
\setlength{\footnotesep}{0.4cm}
\setlength{\skip\footins}{1.1cm}
\usepackage{ragged2e}
\usepackage{scrextend}
\deffootnotemark{\textsuperscript{[\thefootnotemark]}}
\deffootnote[3em]{3em}{0em}{\parbox[b][\height][r]{2.3em}{\footnotesize\textsuperscript{[\thefootnotemark]}}\enskip}
\usepackage{titlesec}
\titleformat{\chapter}[hang]{\bfseries\sizedieciseis}{\MakeUppercase{}\ \thechapter}{5.0mm}{\sizedieciseis\MakeUppercase}
\titleformat{\section}[hang]{\bfseries\sizecatorce}{\thesection}{5.0mm}{\sizecatorce\MakeUppercase}
\titleformat{\subsection}[hang]{\bfseries\sizecatorce}{\thesubsection}{5.0mm}{\sizecatorce}
\titleformat{\subsubsection}[hang]{\bfseries\sizedoce}{\thesubsubsection}{5.0mm}{\sizedoce}
\titleformat{\paragraph}[hang]{\em\sizedoce}{\theparagraph}{5.0mm}{\sizedoce}
\newcommand{\titulos}{\sf\bf\sizecatorce\centerline}
\newcommand{\titulosizq}{\sf\bf\sizecatorce}
\setcounter{secnumdepth}{3}
\setcounter{tocdepth}{2}
\usepackage{float} 
\usepackage[breaklinks]{hyperref} 
\hypersetup{pdfauthor={Luis Miguel Pilamunga Agualsaca},pdftitle={Herramientas de Seguridad Ofensiva},pdfsubject={Evaluación de la Detección de Ataques de Fuerza Bruta Generados por IA con SNORT},pdfkeywords={Seguridad, Ofensiva, IA, SNORT},colorlinks,citecolor=black,filecolor=black,linkcolor=black,urlcolor=black}
\usepackage[all]{hypcap}
\usepackage[titles]{tocloft}
\setlength\cftbeforetoctitleskip{0pt}
\setlength\cftaftertoctitleskip{1cm}
\renewcommand\cftchappresnum{\chaptername\space}
\renewcommand\cftchappresnum{ }
\setlength{\cftchapnumwidth}{2em}
\newcommand\centrarcelda[1]{\let\temp=\\#1\let\\=\temp}
\usepackage{array}
\usepackage{calc}
\usepackage[table]{xcolor}
\usepackage{booktabs}
\usepackage{tabulary}
\usepackage{longtable}
\usepackage{setspace}
\setlength\tymin{5cm}
\usepackage{multirow}
\definecolor{bluetable}{RGB}{175,198,233}
\arrayrulecolor{bluetable}
\setlength{\arrayrulewidth}{.9pt}
\usepackage{hhline}
\usepackage{etoolbox}
\AtBeginEnvironment{longtable}{\small}
\AtBeginEnvironment{tabular}{\small}
\usepackage{pdflscape}
\usepackage{listings}
\usepackage{color}
\definecolor{letraAzul}{cmyk}{1,0.5,0,0.5}
\definecolor{lstrule}{RGB}{158,180,204}
\definecolor{fondo}{RGB}{245,245,250}
\definecolor{gray}{rgb}{0.5,0.5,0.5}
\definecolor{darkviolet}{rgb}{0.5,0,0.4}
\definecolor{darkpink}{rgb}{0.8,0.3,0.5}
\lstset{language=Java,basicstyle=\footnotesize\color{letraAzul},numbers=left,numberstyle=\scriptsize\color{gray},numberfirstline=true,firstnumber=1,stepnumber=5,numbersep=8pt,backgroundcolor=\color{fondo},showspaces=false,showstringspaces=false,showtabs=false,frame=single,rulecolor=\color{lstrule},tabsize=4,captionpos=t,breaklines=true,breakatwhitespace=false,title=\lstname,keywordstyle=\bfseries\color{darkviolet},commentstyle=\color{gray},stringstyle=\color{darkpink},escapeinside={\%*}{*)},morekeywords={*,...},inputencoding=utf8,emphstyle=\color{red},extendedchars=true,literate={á}{{\'a}}1{é}{{\'e}}1{í}{{\'i}}1{ó}{{\'o}}1{ú}{{\'u}}1{ñ}{{\~n}}1}
\usepackage{fancybox}
\usepackage{graphicx,type1cm,eso-pic}
\usepackage[font=small,format=plain,labelfont=bf,up,justification=default,compatibility=false]{caption}
\usepackage{subcaption}
\usepackage{datatool}
\usepackage[toc,acronym,xindy]{glossaries}
\usepackage{mfirstuc}
\renewcommand{\glsnamefont}[1]{\makefirstuc{#1}}
\usepackage{glossary-super}
\usepackage{enumitem}
\newcommand\litem[1]{\item{\bfseries #1\enspace}}
\newenvironment{myindent}[1]{\begin{list}{}{\setlength{\leftmargin}{#1}}\item[]}{\end{list}}
\usepackage{pifont}
\renewcommand{\labelitemi}{\ding{112}}
\renewcommand{\labelitemii}{\ding{71}}
\newcommand{\fullrefuno}[1]{(véase \ref{#1}, pág. \pageref{#1})}
\newcommand{\fullref}[2]{en \ref{#1}, pág. \pageref{#2}}
\newcommand{\refdos}[2]{(véase \ref{#1} y \ref{#2})}
\newcommand{\fullreffig}[1]{Fig.~\ref{#1} pág. \pageref{#1}} 
\newcommand{\fullreftab}[1]{Tab.~\ref{#1} pág. \pageref{#1}} 
\newcommand{\fullrefcod}[1]{Cód.~\ref{#1} pág. \pageref{#1}} 
\newcommand{\fullrefanx}[1]{(véase el Anexo~\ref{#1})} 
\hyphenation{sa-ffer soft-ware Jesse Nielsen}
\usepackage[toc,title,header]{appendix}
\renewcommand{\appendixname}{Anexo}
\renewcommand{\spanishappendixname}{Anexo}
\renewcommand{\appendixtocname}{Anexos}
\renewcommand{\appendixpagename}{Anexos}
\makeatletter
\newcommand*\updatechaptername{\addtocontents{toc}{\protect\renewcommand*\protect\cftchappresnum{\@chapapp\ }\setlength{\cftchapnumwidth}{5em}}}
\makeatother
\makeatletter
\appto{\appendices}{\def\Hy@chapapp{Appendix}}
\makeatother
